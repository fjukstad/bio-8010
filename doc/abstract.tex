\documentclass[11pt,journal,compsoc]{IEEEtran}
\usepackage[utf8]{inputenc}
\usepackage{hyperref} 
\begin{document}
\title{Code Lab}
\author{Bjørn~Fjukstad \\ BIO-8010 Communicating Science Module 3\\ Visualizing
your science} 
\maketitle
\vspace{-15mm}

\section{Introduction} 
Programming is introduced this fall as a mandatory course in all primary and
secondary schools in Great Britain. In countries such as Estonia programming is
already a part of the curriculum, and by 2016 Finland has plans to introduce it
as well. Through foundations such as Code.org in the US, schools are offered
free courses in programming. In the list of backers in the Code.org foundation
we find Bill Gates the founder of Microsoft and Mark Zuckerberg the CEO of
Facebook. Unfortunately there are no plans of introducing programming in the
curriculums in schools in Norway. Lær kidsa koding (LKK) is volunteer
organization that wants to help schools in introducing programming as a part of
their teaching. LKK has teaching material in Norwegian that can be used by
educators to teach children the fundamentals of computer programming. 

For teaching programming to children there exists a plethora of different
systems and tools. For the youngest kids, a popular alternative is
Scratch\footnote{\url{scratch.mit.edu}}, where kids use a visual programming
language where they can make games and other small projects.  For older kids a
popular choice is to go into game modding, specifically modding the popular
video game MineCraft\footnote{\url{minecraft.net}}.
CodeCombat\footnote{\url{codecombat.com}} is another alternative where kids
program game characters through labyrints or different set of tasks. If kids
want a more hands-on approach it is popular to program either Lego
Mindstorms\footnote{\url{mindstorms.lego.com}} or small Arduino
computers\footnote{\url{arduino.cc}}. 

\section{Code Lab}
Code Lab is a game where kids collaborate on navigating a character 
through labyrinths, solving puzzles and fighting enemies. It runs in a shared
environment where kids physically interact with the game by either writing
real code or moving visual code blocks. Code lab consists of two types of 
devices, one large display to run the game on, and one or more input devices for
kids to interact with the game. 

Since the kids need to program the characters to perform different tasks, they
will have to learn the basics of programming. The different levels will require
them to learn about \emph{variables}, \emph{functions} and \emph{control
statements} such as \emph{for}-loops and \emph{if}-statements. 

\subsection{The Tromsø Display Wall}
Since I am a Ph. D. Student in the High-Performance Distributed Systems (HPDS)
group at the Department of Computer Science, I plan on developing the game for
this course. I plan on using the Tromsø Display Wall lab for the shared
environment. 

The Tromsø Display Wall consists of 28 computers each connected to a projector.
These projectors are mounted behind a canvas, projecting their content onto this
canvas. Together the projectors make up a display that has a resolution of 
7168x3072, and a physical size of 6x2m. The display wall is used for recruitment
of high-school students, but hopefully if this game is completed also in the
local teaching activities for the local Kodeklubb coordinated by LKK
Tromsø\footnote{\url{facebook.com/LKK.Tromso}}. 





\end{document} 
