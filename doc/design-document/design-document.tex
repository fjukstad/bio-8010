\documentclass[11pt,journal,compsoc]{IEEEtran}
\usepackage[utf8]{inputenc}
\usepackage{hyperref} 
\begin{document}
\title{Code Lab \\ Design Document}
\author{Bjørn~Fjukstad \\ BIO-8010 Communicating Science Module 3\\ Visualizing
your science} 
\maketitle
\vspace{-15mm}
\section{Introduction} 
Code Lab is a game where kids collaborate on navigating a character 
through labyrinths, solving puzzles and fighting enemies. It runs in a shared
environment where kids physically interact with the game by either writing
real code or moving visual code blocks. Code lab consists of two types of 
devices, one large display to run the game on, and one or more input devices for
kids to interact with the game. 

Since the kids need to program the characters to perform different tasks, they
will have to learn the basics of programming. The different levels will require
them to learn about \emph{variables}, \emph{functions} and \emph{control
statements} such as \emph{for}-loops and \emph{if}-statements. 



\end{document} 
