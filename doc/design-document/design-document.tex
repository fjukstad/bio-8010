\documentclass[12pt,journal,compsoc]{IEEEtran}
\usepackage[utf8]{inputenc}
\usepackage{hyperref} 
\usepackage{fancyhdr}


\begin{document}

\title{Code Lab \\ Design Document}

\author{Bjørn~Fjukstad \\ BIO-8010 Communicating Science Module 3\\ Visualizing
your science} 

\maketitle
\vspace{-15mm}

\section{Introduction} 
Code Lab is a game where kids collaborate on escaping from an underground
dungeon by programming their in-game characters to fight monsters, solve puzzles
and collect gems. The game is played in a collaborative environment such as the
Tromsø Display Wall\cite{anshus2013nineyears}, where kids program on their own
devices and run the game on the large display. The display wall environment
provides an interactive arena where kids can collaborate on completing the game
together.

Since the kids need to program the characters to perform different tasks, they
will have to learn the basics of programming. The different levels will require
them to learn about \emph{variables}, \emph{data structures}, \emph{functions}
and \emph{control statements} such as \emph{for}-loops and \emph{if}-statements. 
As the kids play the game, the puzzles and problems they are faced with will
increase in difficulty, making it necessary to design and implement more
complex solutions. 

The game is intended for children 10 - 16 years old, who already have some
experience with graphical programming environments such as
Scratch\cite{resnick2009scratch}. It is intended for kids that want to learn
more about programming, specifically getting started with text-based
programming. 



% Structure
%\subsection{Concept, Goal and Learning Goals} 
%\subsection{Target Audience} 

\section{Description} 
\subsection{Game Narrative} 
\subsection{Game Setting} 
\subsection{Game Tasks} 

\section{Key features} 
\subsection{Game Mechanics} 
\subsection{Progression} 
\subsection{Reward and Motivation} 
\subsection{Balancing} 

\section{Platform} 
\subsection{Art} 
\subsection{Music and Audio}

\section{Production and Team} 
\section{Competition and Inspiration}


% Fetch your references, in a file called report.bib
\bibliography{design-document}{}
\bibliographystyle{plain}
\end{document} 
