\documentclass[12pt,journal,compsoc]{IEEEtran}
\usepackage[utf8]{inputenc}
\usepackage{hyperref} 
\usepackage{fancyhdr}


\begin{document}

\title{Code Lab \\ Design Document}

\author{Bjørn~Fjukstad \\ BIO-8010 Communicating Science Module 3\\ Visualizing
your science} 

\maketitle
\vspace{-15mm}

\section{Introduction} 
Code Lab is a game where kids collaborate on escaping an underground dungedon by
programming their in-game characters to fight monsters, solve puzzles and
collecting gems. The game is played in a collaborative environment such as the
Tromsø Display Wall\cite{anshus2013nineyears}, where kids program on their own
devices and run the game on the large display. This provides an interactive
arena where kids can collaborate on completing the game together.

Since the kids need to program the characters to perform different tasks, they
will have to learn the basics of programming. The different levels will require
them to learn about \emph{variables}, \emph{functions} and \emph{control
statements} such as \emph{for}-loops and \emph{if}-statements. 


% Fetch your references, in a file called report.bib
\bibliography{design-document}{}
\bibliographystyle{plain}
\end{document} 
