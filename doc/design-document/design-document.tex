\documentclass[12pt,journal,compsoc]{IEEEtran}
\usepackage[utf8]{inputenc}
\usepackage{hyperref} 
\usepackage{fancyhdr}

\usepackage{graphicx}



\begin{document}

\title{Game Title \\ Design Document}

\author{Bjørn~Fjukstad \\ BIO-8010 Communicating Science Module 3\\ Visualizing
your science \\ \url{github.com/fjukstad/bio-8010}} 

\maketitle
\vspace{-15mm}

\section{Introduction} 
Code Lab is a game where kids collaborate on escaping from an underground
dungeon by programming their in-game characters to fight monsters, solve puzzles
and collect gems. The game is played in a collaborative environment such as the
Tromsø Display Wall\cite{anshus2013nineyears}, where kids program on their own
devices and run the game on a large shared display. The display wall environment
provides an interactive arena where kids can collaborate on completing the game
together.

Since the kids need to program the characters to perform different tasks, they
will have to learn the basics of programming. The different levels will require
them to learn about \emph{variables}, \emph{data structures}, \emph{functions}
and \emph{control statements} such as \emph{for}-loops and \emph{if}-statements. 
As the kids play the game, the puzzles and problems they are faced with will
increase in difficulty, making it necessary to design and implement more
complex solutions. 

The game is intended for children 10 - 16 years old, who already have some
experience with graphical programming environments such as
Scratch\cite{resnick2009scratch}. It is intended for kids that want to learn
more about programming, specifically getting started with text-based
programming. 

CodeLab is open-sourced at \url{github.com/fjukstad/bio-8010} and there is a
playable prototype of the first-level at \url{fjukstad.github.io/bio-8010}. 

% Structure
%\subsection{Concept, Goal and Learning Goals} 
%\subsection{Target Audience} 

\section{Background} 
For teaching programming to children there exists a plethora of different
systems and tools. For the youngest kids, a popular alternative is
Scratch\footnote{\url{scratch.mit.edu}}, where kids use a visual programming
language where they can make games and other small projects.  For older kids a
popular choice is to go into game modding, specifically modding the popular
video game MineCraft\footnote{\url{minecraft.net}}.
CodeCombat\footnote{\url{codecombat.com}} is another alternative where kids
program game characters through labyrints or different set of tasks. If kids
want a more hands-on approach it is popular to program either Lego
Mindstorms\footnote{\url{mindstorms.lego.com}} or small Arduino
computers\footnote{\url{arduino.cc}}. 

CodeLab is a game that tries to make learning text-based programming more fun
and collaborative through a video game that kids play in an shared environment.
It takes the gameplay from CodeCombat and the physical hands-on interaction from
Lego Mindstorms and Arduinos, making an interactive and collaborative learning
environment for programming. 

\section{Description} 
CodeLab takes place in a fictional dungeon, where each player is assigned a
hero that he or she controls by programming their actions. The players equip
their heroes with armor, weapons and other items that can help them complete the
different levels. For each level, the players have to complete a set of tasks by
programming their characters by using a programming language similar to the 
Lua programming language\footnote{\url{lua.org}}. Players 
write the code on their local machine, be it a laptop or a smart phone, and see
their characters perform the actions on a large shared display. Alongside the
game view, the players see each others code making it possible to help out
eachother if they encounter any problems. 

The game is turn-based in the sense that every turn is a line of code or the
execution of a program. The players can either run their entire program to pass
the level, or they can interactively type commands that completes the level.
Typically the first levels where the players would only move a character is
suitable for an interactive solution, while later levels require more complex
programs that are difficutly to complete by writing only single lines at a time. 

\subsection{Game Narrative} 
The game starts with the players being introduced to their heroes and the
dungeon they find themselves in. The players get to know what they can equip
their heroes with and the actions they can make their heroes do. Each set of
item has special actions that the players can use, e.g. you need to equip a hero
with boots if you want to move. Some boots give you the power to walk faster,
while some give you the ability to climb obstacles. The players are also
introduced to the programming that they need to do to complete every level. The
first level introduces them to the four commands: $moveDown$, $moveUp$,
$moveRight$ and $moveLeft$ that moves their character either down, up, right or
left. Figure 1 shows a sketch of the first level. The player is the blue
circle, and the goal of the level is to navigate down to the yellow square.
In the full game you would image that the circle is replaced with a more
graphical game character, and that the yellow square could be a person that
needs help or assistance. 

\begin{figure}[htb]
    \begin{centering}
    \includegraphics[width=0.47\textwidth]{./figures/codelab3.png}
    \caption{A sketch of the first level of CodeLab. The goal of the level
    is to write code that moves the character (the circle) down to the
    yellow square. The player writes three commands, $moveDown$,
    $moveRight$ and $moveDown$ to complete the level. } 
    \label{fig:level1}
    \end{centering} 
\end{figure}


In the first stages of the game the players would also need to familiarize
themselves in the game setting. CodeLab is designed to run in a collaborative
environment such as the Tromsø Display Wall, where each player writes code on
their own device while the graphical output of game is shown on a large shared
display. This encourages collaboration between the players, and creates a more
interactive learning environment for the children playing the game. From my
experience with the local Code Club, having something run in a shared setting
makes the whole coding experience more fun and collaborative than working on
your own computer. 

As the game progresses both the difficulty of the levels and the complexity of
the code the players produce increase. The first levels will concentrate on the
basics of programming, with different commands that the players can execute.
Following these the players are introduced to \emph{conditionals} such as
\emph{if}-statements or \emph{for}-loops, \emph{data structures} and
\emph{variables}, and \emph{functions}. As the game progresses the players will
have access to more commands and features, such as the ability to fight
monsters, build structures and forge weapons. 

The ultimate goal of the game is to escape the dungeon that the player is
trapped in, and have learned the basic skills needed to get started with
text-based programming. 

\subsection{Game Environment} 
Figure \ref{fig:environ} shows an illustration of the shared environment CodeLab
I envision children playing the game in. Kids collaborate on writing the code
that makes it possible to complete a level. They can write the code on their own
devices, but must run the game on the shared display where other kids can view
their progress and code. 

\begin{figure}[htb]
    \begin{centering}
    \includegraphics[width=0.4\textwidth]{./figures/codelab2.png}
    \caption{The CodeLab environment. Players collaborate to solve a level. Both
    the graphical window where the game runs, as well as the source code is
    shown on a large display. } 
    \label{fig:environ}
    \end{centering} 
\end{figure}

In addition to just having one game session on the shared display, some levels
of CodeLab requires that kids collaborate to continue to the next level. Figure
\ref{fig:2p} illustrates how players are given similar levels that they have to
complete together to proceed.  Collaborating on the different levels will help
the kids learn more about the collaborative side of programming by sharing code
and helping others. 

\begin{figure}[htb]
    \begin{centering}
    \includegraphics[width=0.47\textwidth]{./figures/codelab5.jpg}
    \caption{Two players trying to complete the first level. This view is shown
    on the large display, where they can help each other write the necessary
    code. } 
    \label{fig:2p}
    \end{centering} 
\end{figure}

\subsection{Game Tasks} 
The overall goal of CodeLab is to escape the dungeon that our heres has fallen
into, and by doing so learning text-based programming. The game itself is broken
into smaller levels that the players need to complete by programming the actions
of their characters. Every level will have one or more specific goals that the
game characters have to complete, e.g. defeat an enemy or get to a specific
location. Some levels will also require that the players use specific
programming skills, such as placing actions within a \emph{for}-loop or a
\emph{function}. 


\section{Game Mechanics} 
CodeLab is in essence two games in one. The first being the actual dungeon where
the heroes live, where they must fight enemies and escape. The other is the
coding-part where kids use their skills to control their heroes through
real programming. These are tightly coupled to each other, new items allow the
players to use new functions and so on. 

The game within code lab is a role playing game where the characters wear armor
and fight enemies with swords, spears and other medieval weapons. 


\subsection{Progression} 

\begin{figure}[htb]
    \begin{centering}
    \includegraphics[width=0.4\textwidth]{./figures/codelab4.jpg}
    \caption{A level that can be completed by writing a simple loop.} 
    \label{fig:loop}
    \end{centering} 
\end{figure}


\subsection{Reward and Motivation} 
\subsection{Balancing} 

\section{Platform} 
\subsection{Art} 
\subsection{Music and Audio}

\section{Production and Team} 
\section{Competition and Inspiration}


% Fetch your references, in a file called report.bib
\bibliography{design-document}{}
\bibliographystyle{plain}
\end{document} 
